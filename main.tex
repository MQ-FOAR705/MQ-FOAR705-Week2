% Latex template: https://github.com/mqTeXUsers/Macquarie-University-Beamer-Theme

% Slide Masters:

% Title
% Text
% 2 column
% Full-image
% Bibliography
% Closing
 
\documentclass[aspectratio=43, 11pt]{beamer} % Aspect ratio
% https://tex.stackexchange.com/a/14339/5483 
% Possible values: 1610, 169, 149, 54, 43 and 32.
% 169 = 16:9

\PassOptionsToPackage{table}{xcolor}    %https://tex.stackexchange.com/a/5365/5483

\usetheme{macquarie}
\usepackage{multicol} % https://tex.stackexchange.com/a/396018/5483
\usepackage{xurl}
\usepackage[british]{babel}       % Set language
% \usepackage[utf8x]{inputenc}      % Set encoding
\usepackage{colortbl}
\mode<presentation>           % Set options
{
  \usetheme{default}          % Set theme
  \usecolortheme{default}         % Set colors
  \usefonttheme{default}          % Set font theme
  \setbeamertemplate{caption}[numbered] % Set caption to be numbered
}

% Uncomment this to have the outline at the beginning of each section highlighted.
%\AtBeginSection[]
%{
%  \begin{frame}{Outline}
%    \tableofcontents[currentsection]
%  \end{frame}
%}

\usepackage{graphicx}         % For including figures
\usepackage{booktabs}         % For table rules
\usepackage{hyperref}         % For cross-referencing


\usepackage{enumitem} % https://tex.stackexchange.com/a/2292/5483

%https://tex.stackexchange.com/a/371844/5483
\setbeamerfont{bibliography entry author}{size=\tiny}
\setbeamerfont{bibliography entry title}{size=\tiny}
\setbeamerfont{bibliography entry location}{size=\tiny}
\setbeamerfont{bibliography entry note}{size=\tiny}
\setbeamerfont{bibliography item}{size=\tiny}

%https://tex.stackexchange.com/q/333587/5483
%TODO SHAWN REPLACE OSF URL
%\setbeamertemplate{footline}{\strut~\texttt{https://github.com/MQ-FOAR705/MQ-FOAR705-Week1}\hfill\insertframenumber~/~\inserttotalframenumber\strut~~~}

\title{FOAR705 Week 2} % Presentation title
\author{Brian Ballsun-Stanton | Shawn A Ross | Kathryn Elliot}               % Presentation author
\institute{Faculty of Arts}         % Author affiliation
\date{Friday 09 August 2019}                 % Today's date  
\begin{document}

% Title page
% This page includes the informations defined earlier including title, author/s, affiliation/s and the date
% \begin{frame}[noframenumbering]

\maketitle

  
% \end{frame}

\begin{frame}{Today's Plan}
  \tableofcontents
\end{frame}


\section{The Learning Journal}

\begin{frame}{Learning Journals}
\begin{itemize}[label=\textbullet]
    \item To serve as a Laboratory Notebook -- a record of:
    \begin{itemize}[label=\textbullet]
        \item Thoughts
        \item Intentions
        \item Results
    \end{itemize}
    \item Mechanism for showing your work
    \item Reminders of common mistakes and solutions
    \item Demonstration of growth over time
\end{itemize}    
\end{frame}

\begin{frame}{Grading Overview}
Percentages:
\begin{itemize}[label=\textbullet]
    \item Exercise Documentation (70)
    \item Committing your work (10)
    \item Error Reflection and Solution (20)
\end{itemize}    

Dates:
\begin{itemize}[label=\textbullet]
%(Weeks 4, 6, 8, 13) 
    \item Spreadsheets Learning Journal (with all exercises complete) due before class, 23 August

    \item Shell Scripts learning Journal due (with all exercises) before class, 6 September

    \item Open Refine learning Journal due (with all exercises) before class, 4 October
    \item R Learning Journal Due (up to dplyr and tidyr) before class, 8 November

\end{itemize}    

\end{frame}

\begin{frame}{Learning Journal pattern}

All technical work outside of class, and all Carpentry exercises should be recorded in a document in cloudstor.

For each discrete action taken. (Exercise, part of exercise, command run, code changed, code added, etc...)
\begin{itemize}[label=\textbullet]
    \item The intention of the action: ``What do you intend to be the result of the action you are about to take''
    \item The specifics of the action taken: Timestamp, commands or actions
    \item Results: note success or failure, screenshot or copy-paste or summary, error states
    \item Marginal notes for improvements on what to do or how to think about the idea more effectively
\end{itemize}

\textbf{Documentation of errors is critical for success}
% , , and the results, along with any marginal notes for improvement or updating your mental model of what should have had happened
% All technical work outside of class should be recorded in a laboratory notebook (document kept with the code or on cloudstor) which documents the intention of the action, the specifics of the action taken, and the results, along with any marginal notes for improvement or updating your mental model of what should have had happened. This documentation includes: an answer to the specific objective: "What do you intend to be the result of the action you are about to take", the action to be taken containing timestamp, and commands or actions performed, and the result, documenting what happened, success or failure in relation to the objective, and error states. Documentation of errors (each in its own entry) and their remediation are strongly encouraged.


\end{frame}

\begin{frame}{Example}

\begin{figure}[H]
        \centering
        \includegraphics[height=.75\textheight]{figures/anaScreenshot.png}
        \caption{Screenshot from a Software Carpentry Learning Journal}
        \label{fig:screenshotAna}
    \end{figure}
    
\end{frame}

\begin{frame}{What is an HD?}
\begin{itemize}[label=\textbullet]

\item Errors are documented along with steps to recover, just like any other action in the lab notebook. 
\item All entries have an objective articulated before they are run. 
\item Commands and code are clearly indicated for future reference. 
\item Results are clearly documented, and contain a minimal amount of self-reflective feedback to inform the next attempt.
\end{itemize}

\end{frame}

\begin{frame}{Committing code and exercises}

For Carpentries exercises or other experiments (code not directly related to your proof of concept), make a repository (public or private) inside the MQ-FOAR705 organisation on Github. Commit code or outputs (reorganised sheets, etc) to that repository.

An HD:
\begin{itemize}[label=\textbullet]
\item Commit messages are clear and have useful descriptions, not just summary lines. 
\item Files and directories are organised consistently in a fashion which allows for easy command line navigation and sorting (no spaces or other problematic characters). 
\item All work, not just technical, is committed to an appropriate repository. 
\end{itemize}

% , , and the results, along with any marginal notes for improvement or updating your mental model of what should have had happened
% All technical work outside of class should be recorded in a laboratory notebook (document kept with the code or on cloudstor) which documents the intention of the action, the specifics of the action taken, and the results, along with any marginal notes for improvement or updating your mental model of what should have had happened. This documentation includes: an answer to the specific objective: "What do you intend to be the result of the action you are about to take", the action to be taken containing timestamp, and commands or actions performed, and the result, documenting what happened, success or failure in relation to the objective, and error states. Documentation of errors (each in its own entry) and their remediation are strongly encouraged.


\end{frame}
\begin{frame}{Error Reflection and Solutions}
Document your errors and how you've found solutions. 

An HD:
\begin{itemize}[label=\textbullet]
\item Using the error documentation in the labratory notebook as the basis, a library of common errors, solutions, and ways to find solutions is built for future reference. 
\item These solutions contain links to good internet resources, and useful rules of thumb for where best to find assistance.
\end{itemize}
\end{frame}

\section{Data Carpentry}
\begin{frame}{Shared notes}

Shared notes document for the entire unit. Put questions, thoughts or observations from readings or prior class onto page 2. The person who isn't lecturing will try to answer them inline.



    \begin{figure}[H]
        \centering
        \includegraphics[height=.6\textheight]{figures/cloudstorqr.png}
        \caption{Cloudstor: \url{http://bit.ly/2YW1Owm}}
        \label{fig:programmed}
    \end{figure}


\end{frame}


\begin{frame}{Introduction}

Guiding question for this episode: 

What are basic principles for using spreadsheets for good data organisation?


\end{frame}
\begin{frame}{Introduction - Exercise}

In cloudstor shared document:
\begin{itemize}[label=\textbullet]
\item How many people have used spreadsheets in their research?
\item How many people have accidentally done something that made them frustrated or sad?
\end{itemize}

\end{frame}

\begin{frame}{Formatting data tables in Spreadsheets}
\begin{columns}
\begin{column}{0.5\textwidth}
\begin{figure}[H]
        \centering
        \includegraphics[height=.6\textheight]{figures/multiple-info.png}
        \caption{Data Carpentry: combined info (CC-BY)}
        \label{fig:combined}
    \end{figure}
    \end{column}
    \begin{column}{0.5\textwidth}

    \begin{figure}[H]
    \begin{center}
        \includegraphics[height=.6\textheight]{figures/single-info.png}
        \caption{Data Carpentry: single info (CC-BY)}
        \label{fig:single}
        \end{center}
    \end{figure}
    \end{column}
    \end{columns}
\end{frame}




\begin{frame}{Exercise}

We’re going to take a messy version of the SAFI data and describe how we would clean it up.

\begin{itemize}[label=\textbullet]

\item Download the messy data.
\item Open up the data in a spreadsheet program.
\item Notice that there are two tabs. Two researchers conducted the interviews, one in Mozambique and the other in Tanzania. They both structured their data tables in a different way. Now, you’re the person in charge of this project and you want to be able to start analyzing the data.
\item With the person next to you, identify what is wrong with this spreadsheet. Discuss the steps you would need to take to clean up the two tabs, and to put them all together in one spreadsheet.
\item Document your group's thoughts in the cloudstor shared document.
\end{itemize}

\textbf{Important} Do not forget our first piece of advice, to create a new file (or tab) for the cleaned data, never modify your original (raw) data.

After you go through this exercise, we’ll discuss as a group what was wrong with this data and how you would fix it.

\end{frame}

\begin{frame}{Metadata}

``Data about data''

Exercise (maybe):

Download a clean version of this dataset and open the file with your spreadsheet program. This data has many more variables that were not included in the messy spreadsheet and is formatted according to tidy data principles.

Discuss this data with a partner and make a list of some of the types of metadata that should be recorded about this dataset. It may be helpful to start by asking yourself, “What is not immediately obvious to me about this data? What questions would I need to know the answers to in order to analyze and interpret this data?”


\end{frame}



\begin{frame}{Git clients}
\begin{itemize}[label=\textbullet]
\item Web interface (demonstrate now)
\item desktop.github.com client
\item other clients
\end{itemize}

Exercise: commit the results (a text file containing your thoughts and the original data) of the cleanup exercise to your own repository on Github now.

\end{frame}

\begin{frame}{Homework}

Before class next week, in your learning journal, finish reading to ``Formatting Problems'' and document the two exercises we've (hopefully) done today in your Github repository. Also in your learning journal, find an example of each problem in data produced by your discipline. (Bonus points if you can find these problems in published datasets in your discipline).

\end{frame}

\begin{frame}{Breaktime}
5 minute break
\end{frame}

\section{Research design 101}

\begin{frame}{Articulating research design}
Robust research requires articulation of an explicit research design 
    \begin{itemize}[label=\textbullet]
        \item Deductive vs. Inductive vs. Abductive
        \item Idiographic vs. Nomothetic 
    \end{itemize}   
Each type is valuable, but you must recognise what you are doing and not conflate them.
\end{frame}

\section{Proof of Concept scoping 101}

\begin{frame}{What is scoping?}
To ensure that a product actually solves user problems, software development begins with `Business Analysis' (BA), where user requirements are enumerated. 

But how do we gather requirements and act on them to produce a solution?

It is surprisingly difficult to learn what clients (even yourself!) really need and want
\end{frame}

\begin{frame}{Approach \#1: Ask what people want (opinions)}
 \begin{figure}[Top-down-planning]
    \centering
        \includegraphics[height=.75\textheight]{figures/Archaeologists-standards.png}
        \caption{Archaeologists contemplate data standards (FAIMS Stocktaking, 2012)}
        \label{fig:standards}
    \end{figure}
\end{frame}

\begin{frame}{Approach \#2: Ask or observe what people do (facts)}
    Borrowed from 'Lean startup' methodology. All materials are on Cloudstor or available from \url{https://www.strategyzer.com/} (free registration required)
    
    The overview document is the 'Value Proposition canvas', but see also the 'Mission Model canvas' for an indication of how the approach can be applied outside tech industry settings.
\end{frame}

\begin{frame}{Understand your client}
    See the 'Customer Profile' worksheet
    \begin{itemize}[label=\textbullet]
        \item Identify 'jobs' (see 'A day in the life worksheet')
        \item Identify 'pains' (see 'Customer Pains trigger questions')
        \item Identify 'gains' (see 'Customer Gains trigger questions')
    \end{itemize}
\end{frame}

\begin{frame}{Ideate a solution}
    After you have completed the 'Customer Profile' worksheet, look at the 'Value Map' worksheet
    \begin{itemize}[label=\textbullet]
        \item Identify 'pain relievers' that map to pains (see 'Pain Relievers trigger questions')
        \item Identify 'gains creators' that map to gains (see 'Gain Creator trigger questions')
        \item Finally, you can articulate 'Products and Services' - what you are going to build 
    \end{itemize}
\end{frame}

\section{Project management 101}

\begin{frame}{Developing ideas towards a solution}
    So you have great ideas, what next? 
\end{frame}

\begin{frame}{Approach \#1: Top-down design (`Waterfall')}
 \begin{figure}[Waterfall]
    \centering
        \includegraphics[height=.75\textheight]{figures/waterfall.png}
        \caption{Traditional and linear approach} \cite{Parody2018-if}
        \label{fig:6}
 \end{figure}
\end{frame}

\begin{frame}{Approach \#2: Iterative design with course corrections (`Agile')}
 \begin{figure}[Agile]
    \centering
        \includegraphics[height=.75\textheight]{figures/agile.png}
        \caption{Design-test-repeat approach to PM} \cite{Parody2018-if}
        \label{fig:7}
 \end{figure}
\end{frame}

\begin{frame}{Manifesto for Agile Software Development}
We are uncovering better ways of developing software by doing it and helping others do it. Through this work we have come to value:
    \begin{itemize}[label=\textbullet]
        \item \textbf{Individuals and interactions} over processes and tools
        \item \textbf{Working software} over comprehensive documentation
        \item \textbf{Customer collaboration} over contract negotiation
        \item \textbf{Responding to change} over following a plan
    \end{itemize}
That is, while there is value in the items on the right, we value the items on the left more. \cite{Atlassian2019-xl}
\end{frame}

\begin{frame}{Tool \#1: Gantt chart}
 \begin{figure}[Gantt]
    \centering
        \includegraphics[height=.75\textheight]{figures/gantt.png}
        \caption{A Gantt Chart template (on Cloudstor)}
        \label{fig:8}
 \end{figure}
\end{frame}

\begin{frame}{Tool \#2: Kanban board}
 \begin{figure}[Kanban]
    \centering
        \includegraphics[height=.75\textheight]{figures/kanban.png}
        \caption{Schematic Kanban board \cite{Atlassian2019-bo}}
        \label{fig:9}
 \end{figure}
\end{frame}

\begin{frame}{A Kanban board should}
    \begin{itemize}[label=\textbullet]
        \item Visualise your work
        \item Limit work in progress
    \end{itemize}
Kanban boards often have columns like: Backlog (wish list), To do, In progress, Done, with the 'To do' and 'In progress' columns having work limits (e.g., 3-5 tasks). 

Trello is a popular application for Kanban. Atlassian has Kanban learning materials online \cite{Atlassian2019-bo}.
\end{frame}

% Outline
% This page includes the outline (Table of content) of the presentation. All sections and subsections will appear in the outline by default.
% \begin{frame}{The context of Research Data Management}
%   \tableofcontents
% \end{frame}

% % The following is the most frequently used slide types in beamer
% % The slide structure is as follows:
% %
% %\begin{frame}{<slide-title>}
% % <content>
% %\end{frame}

% \section{Code of Conduct}

% \begin{frame}{Unit Code of Conduct}
% This class is using a great deal of material from The Carpentries. All interactions related to this class, inside and outside, abide by The Carpentries Code of Conduct.

% Report code of conduct violations to Shawn, Brian, or eresearch@mq.edu.au.

% \url{https://docs.carpentries.org/topic_folders/policies/code-of-conduct.html}

% In summary, we want to emphasise:

% \begin{itemize}[label=\textbullet]
%     \item Use welcoming and inclusive language
%     \item Be respectful of different viewpoints and experiences
%     \item Gracefully accept constructive criticism
%     \item Focus on what is best for the community
%     \item Show courtesy and respect towards other community members
% \end{itemize}

% \end{frame}

% \section{Expectations}

% \begin{frame}{Is the content 'too hard'?}
%  `I still have my concerns about how over-technical this course is given it is now meant to be taken by students from across the entire Faculty from diverse backgrounds and with diverse interests...I suspect will cause students anxiety and maybe lead to drop out.'
%     \begin{itemize}[label=\textbullet]
%         \item Before we start, what was your reaction to reading the Unit description?
%         \item Do you agree with the quote above?
%     \end{itemize}
% \end{frame}

% \begin{frame}{Expectations and workload}
%   You are undertaking an Masters of Research at a top one-percent university (QS ranking 125 in Arts and Humanities, 202 in Social Sciences). Expectations and workload higher than what you are accustomed to.
%     \begin{itemize}[label=\textbullet]
%         \item Expect a workload of six hours per week outside of class to earn a DN or HD.
%         \item Avoid missing classes. If you do, expect to spend four hours to catch up.
%         \item If you want to continue to a PhD you need to maintain a DN or HD average.
%         \item Both of us have taught overseas and are engaged with international trends in research technology. This unit has been calibrated to the international environment.
%         \item Considering the academic job market, competition is fierce.
%         \item Most of you will not get academic jobs, so transferable skills are crucial.
%         \item It is our job to prepare you for this environment, and yours to make yourself competitive.
%     \end{itemize}
% \end{frame}

% \begin{frame}{Assessment}

% \begin{itemize}[label=\textbullet]
%     \item Proof of Concept
%     \item Original Software Publication
%     \item Lightning talk
%     \item Learning journal
% \end{itemize}

% \end{frame}

% \section{Don't panic!}

% \begin{frame}{Data Carpentry: a proven approach}
%     `Building communities teaching universal data literacy'
       
%     `Data Carpentry trains researchers in the core data skills for efficient, shareable, and reproducible research practices. We run accessible, inclusive training workshops; teach openly available, high-quality, domain-tailored lessons; and foster an active, inclusive, diverse instructor community that promotes and models reproducible research as a community norm.' \cite{Teal2016-gy}

%     `Since 1998, Software Carpentry has been teaching researchers the computing skills they need to get more done in less time and with less pain. Our volunteer instructors have run hundreds of events for more than 34,000 researchers since 2012.' \cite{Duckles2018-fu}
% \end{frame}

% \begin{frame}{Data Carpentry: widely used worldwide in HASS}
%     Carpentries training is used all over the world to teach digital literacy and computational thinking to Humanities and Social Sciences students and researchers.
%     \begin{itemize}[label=\textbullet]
%         \item Digital Humanities at Oxford Summer School
%         \item CODATA-RDA School of Research Data Science
%         \item Australian Research Data Cloud training
%         \item THATCamps (e.g., at Sydney ResBaz 2019)
%     \end{itemize}
% \end{frame}

% \begin{frame}{Data Carpentry: used at Macquarie}
%   Other MRes students at this university have successfully undergone DC training:
%     \begin{itemize}[label=\textbullet]
%         \item BIOL703 Research Skills for Biology
%         \item No excess attrition, high student satisfaction, good feedback
%         \item Nominated for a Vice-Chancellor's Learning and Teaching award
%         \item Is the background or needs of Arts students that different from ecology, biology, environmental sciences, and related fields?
%     \end{itemize}
% \end{frame}

% \begin{frame}{Previous HASS MRes students have thrived}
%   \url{https://www.youtube.com/watch?v=r9jpe9_2z3c}
% \end{frame}

% \section{What, and why?}

% \begin{frame}{Digital literacy: creators, not consumers}
%     \begin{figure}[H]
%         \centering
%         \includegraphics[height=.6\textheight]{figures/2011-ProgOrBeProgged-248x340.jpg}
%         \caption{Program or be Programmed, Douglas Rushkoff}
%         \label{fig:programmed}
%     \end{figure}
  
%   See also: \url{https://impossiblehq.com/an-unexpected-ass-kicking/}
% %insert 'Program or be programmed' book cover image, and link to 'An unexpected ass kicking' %https://rushkoff.com/books/program-or-be-programmed/
% %https://impossiblehq.com/an-unexpected-ass-kicking/

% \end{frame}

% \begin{frame}{Computational thinking: what can you do with a computer?}
% \begin{figure}[H]
%         \centering
%         \includegraphics[height=.6\textheight]{figures/ctc-w2b.jpg}
%         \caption{'To flourish in today's world, computational thinking has to be a fundamental part of the way people think and understand the world.' \cite{Center_for_Computational_Thinking2012-tt}}
%         \label{fig:ctc}
%     \end{figure}
% %insert https://www.cs.cmu.edu/~CompThink/images/ctc-w2b.jpg
% %caption: 'To flourish in today's world, computational thinking has to be a fundamental part of the way people think and understand the world.' https://www.cs.cmu.edu/~CompThink/

% \end{frame}

% \begin{frame}{Tools and approaches}
% Only within these frameworks can you use available tools and approaches - but we will introduce you to a range of them, customised to the disciplinary mix in the class.
%     \begin{itemize}[label=\textbullet]
%         \item Research design and project management
%         \item Data management planning
%         \item Data capture
%         \item Data analysis and collaboration
%         \item Data archiving and dissemination
%     \end{itemize}

% \end{frame}


% \section{Tools and Communication}
% \begin{frame}{Discussion on which tools we will use as a class}

% \begin{itemize}[label=\textbullet]
%     \item Chat/coordination/project management software
%     \item Typesetting software
%     \item Version control online repository
%     \item File sharing mechanisms
%     \item Backup mechanisms
% \end{itemize}

% \end{frame}

% \begin{frame}{Coordination outside of class}

% \begin{itemize}[label=\textbullet]
%     \item Hacky-hour/study groups: \url{https://science.mozilla.org/programs/studygroups}
%     \item Consultation Hours: Friday 12:45-1:45pm (AHH Level 2 lobby) and 4:15-5:15pm, campus hub (before and after seminar)
%     \item \url{https://twitter.com/Rusers_MQ}
% \end{itemize}

% \end{frame}

% \section{Moving on to Data Carpentry}


% \begin{frame}{Pre-Carpentry survey}

% At the start and end of every carpentries workshop, we poll participants.

% \url{https://bit.ly/FOAR705-pre}

% \begin{figure}[H]
%         \centering
%         \includegraphics[height=.6\textheight]{figures/qr.jpeg}
%         \caption{\url{https://mqedu.qualtrics.com/jfe/form/SV_5v6iQJSBZDNhq4d?workshop=FOAR705-2019}}
%         \label{fig:foarqr}
%     \end{figure}
    
% \end{frame}

% \begin{frame}{Sticky notes}

% We use sticky notes during our workshops (and thus during our classes) to indicate progress or needs for assistance. 

% We also use them as minute cards for feedback and the end of each session. 

% \end{frame}
% \begin{frame}{Starting the workshop}
%     \begin{itemize}
%         \item \url{https://datacarpentry.org/socialsci-workshop/}
%         \item \url{https://datacarpentry.org/spreadsheets-socialsci/setup.html}
%         \item \url{https://datacarpentry.org/openrefine-socialsci/setup.html}
%         \item \url{https://datacarpentry.org/r-socialsci/setup.html}
%     \end{itemize}
% \end{frame}


% % \bibliographystyle{apalike}

% % Adding the option 'allowframebreaks' allows the contents of the slide to be expanded in more than one slide.
% % The "1" comes from the outer theme"


% \section{Minute cards!}
% \begin{frame}{Feedback time}

% On your green sticky, write one thing we did well today.

% On your red sticky, write one thing we could improve upon for next time. Be specific. 

% \end{frame}

\section{References}

\begin{multicols}{2}[]
\bibliography{references}
\bibliographystyle{apalike}
\end{multicols}


% \begin{frame}[allowframebreaks]{References}
  
%   \bibliography{references}
%   \bibliographystyle{apalike}
% \end{frame}


\begin{frame}{Thank you!}

% This presentation is available at:
% \texttt{https://osf.io/...}

Source code for this presentation is available at: \url{https://github.com/MQ-FOAR705/MQ-FOAR705-Week2}

This work is licensed under a Creative Commons Attribution 4.0 International License.

\end{frame}



\end{document}